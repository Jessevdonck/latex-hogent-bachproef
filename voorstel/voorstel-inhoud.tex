%---------- Inleiding ---------------------------------------------------------

% TODO: Is dit voorstel gebaseerd op een paper van Research Methods die je
% vorig jaar hebt ingediend? Heb je daarbij eventueel samengewerkt met een
% andere student?
% Zo ja, haal dan de tekst hieronder uit commentaar en pas aan.

%\paragraph{Opmerking}

% Dit voorstel is gebaseerd op het onderzoeksvoorstel dat werd geschreven in het
% kader van het vak Research Methods dat ik (vorig/dit) academiejaar heb
% uitgewerkt (met medesturent VOORNAAM NAAM als mede-auteur).
% 

\section{Inleiding}%
\label{sec:inleiding}

Het opleiden van verpleegkundigen vereist een hoog niveau van competentie in essentiële medische handelingen, waarbij de veiligheid en het welzijn van de patiënt centraal staan. Binnen het zorgonderwijs vertrouwt men traditioneel op fysieke simulatielabs, maar deze aanpak wordt geconfronteerd met significante uitdagingen. De labs zijn duur in aanschaf en onderhoud, en de beperkte beschikbaarheid resulteert in ontoereikende toegang en strikte planning voor studenten \autocite{Pottle2019}. Traditionele methoden bieden bovendien vaak een onvoldoende realistische context om complexe besluitvorming effectief te trainen en missen efficiënte mechanismen voor de geautomatiseerde validatie van competenties met het oog op permanente educatie \autocite{Radianti2020}.

Extended Reality (XR) biedt een potentieel disruptieve oplossing voor deze problemen. XR-simulaties kunnen een veilige, herhaalbare en hoogst realistische digitale omgeving creëren die de fysieke beperkingen van het traditionele lab omzeilt. Dit maakt onbeperkt oefenen met directe, datagedreven feedback mogelijk. Dit project richt zich op de hogeschool HOGENT met de noodzaak om de vaardigheidstraining te moderniseren en te schalen, inclusief de validatie voor CME-credits (\textit{Continuing Medical Education}).

Deze bachelorproef richt zich op de ontwikkeling van XR SkillSim, een uitbreidbaar XR-platform voor medische vaardigheidstraining. Om de haalbaarheid en meerwaarde van dit platform binnen de scope van een bachelorproef aan te tonen, wordt één concrete handeling uitgewerkt als Proof of Concept, namelijk basisreanimatie. Deze handeling is bijzonder geschikt omdat ze zowel technische vaardigheden als soft skills omvat en sterk afhankelijk is van correcte volgorde, timing en besluitvorming.

Om deze doelstelling te bereiken, zullen in deze bachelorproef de volgende onderzoeksvragen worden uitgewerkt:
\begin{itemize}
    \item Hoe kan een XR-toepassing op Meta Quest 3 realistische medische simulaties ondersteunen die voldoen aan didactische vereisten voor verpleegkunde?
    \item Wat zijn de minimale technische en gebruiksvoorwaarden om een schaalbare en uitbreidbare XR-simulatieomgeving te ontwikkelen?
    \item In welke mate ervaren gebruikers (studenten en docenten) de XR-ervaring als realistisch, nuttig en gebruiksvriendelijk?
    \item Welke kwaliteitscriteria gelden voor CME-credits in de zorgsector, en in hoeverre kan de PoC de kwantitatieve data leveren die met deze criteria in lijn ligt?
    \item Wat zijn de technische en organisatorische stappen om de POC door te ontwikkelen naar een duurzame 'HOGENT XR Academy'?
\end{itemize}
Dit onderzoek zal, door middel van de ontwikkeling van de Proof of Concept en een gedetailleerde architectuuranalyse, een gefundeerde basis bieden voor de integratie van XR als een volwaardig en gevalideerd onderdeel van het zorgonderwijs.
%---------- Stand van zaken ---------------------------------------------------

\section{Literatuurstudie en State-of-the-Art}%
\label{sec:literatuurstudie}

In deze sectie wordt de huidige stand van zaken (\textit{state-of-the-art}) rond de toepassing van Extended Reality (XR) in het zorgonderwijs besproken. De focus ligt op de didactische en technologische context waarbinnen de Proof of Concept (POC) van deze bachelorproef zal worden ontwikkeld.

\subsection{Beperkingen van Traditionele Simulatie}

De traditionele methode voor het aanleren en valideren van verpleegkundige vaardigheden berust op fysieke simulatielabs. Hoewel deze labs essentieel zijn voor de basisvorming, zijn ze tegelijkertijd duur in aanschaf en onderhoud, wat leidt tot logistieke complexiteit en beperkte oefentijd per student \autocite{Pottle2019}. Een cruciaal tekort is dat deze omgevingen vaak een onvoldoende realistische contextuele druk creëren, waardoor de ontwikkeling van klinische besluitvorming en kritisch denken niet optimaal getraind wordt \autocite{Radianti2020}. De noodzaak aan realistische, veilige en herhaalbare omgevingen, zonder risico voor patiëntveiligheid, is de drijvende kracht achter de adoptie van nieuwe technologieën in het onderwijs \autocite{Lampropoulos2025}.

\subsection{Didactische en Klinische Potentie van Extended Reality (XR)}

Extended Reality (XR), waaronder Virtual Reality (VR), stelt gebruikers in staat om een virtuele aanwezigheid te ervaren in computergegenereerde omgevingen. Recent onderzoek bevestigt dat VR een effectief educatief hulpmiddel is dat zowel studenten als zorgprofessionals kan ondersteunen \autocite{Lampropoulos2025}. De immersieve, realistische en veilige omgevingen die XR creëert, stellen studenten in staat om hun kennis, praktijkvaardigheden en besluitvorming te verbeteren zonder patiëntveiligheid in gevaar te brengen \autocite{Lampropoulos2025}. Studies tonen een positieve invloed van VR-training aan op klinische vaardigheden, procedurele training en kennisretentie, vergelijkbaar met, en soms superieur aan, traditionele methoden \autocite{Kyaw2019}. Naast technische vaardigheden, draagt VR ook positief bij aan de affectieve domeinen, zoals het vergroten van empathie en gedragsmatig inzicht bij studenten \autocite{Lampropoulos2025}. 

\subsection{Technologische Implementatie en Schaalbaarheid}

De brede adoptie van immersieve simulaties werd in het verleden belemmerd door hoge hardwarekosten. Met de komst van stand-alone headsets, zoals de Meta Quest 3, is de implementatiedrempel voor het hoger onderwijs aanzienlijk verlaagd \autocite{Pottle2019}. Dit platform levert de nodige balans tussen grafische capaciteit en draagbaarheid om een realistische handeling te simuleren en tegelijkertijd schaalbare adoptie binnen een instelling als HOGENT mogelijk te maken. Voor de doorontwikkeling van de POC naar een 'XR Academy' is een robuuste en modulaire technische architectuur vereist \autocite{Radianti2020}. Deze architectuur moet flexibel zijn voor de integratie van nieuwe scenario's en de gelijktijdige verwerking van de prestatiedata van meerdere gebruikers, wat een direct raakvlak vormt met de technische onderzoeksvragen van deze bachelorproef \autocite{Hung2024}.

\subsection{Validatie en Erkenning (CME-credits)}

Voor de duurzame inzet van het XR-platform is de formele erkenning van de training door de toekenning van Continuing Medical Education (CME) credits cruciaal. Dit vereist dat de evaluatiemethoden voldoen aan strenge, objectieve kwaliteitsnormen. Dit betekent dat de XR-applicatie niet alleen de handeling realistisch moet simuleren, maar ook betrouwbare, kwantificeerbare data moet verzamelen over de prestaties van de gebruiker (fouten, stappen, timing) \autocite{Radianti2020}. Onderzoek naar de ontwikkeling en evaluatie van VR-programma's voor specifieke verpleegkundige training benadrukt de noodzaak van een grondige evaluatie van de haalbaarheid en de impact op de leerresultaten \autocite{Hung2024}. Dit onderzoek richt zich op het vertalen van deze kwaliteitscriteria naar de technische en pedagogische functionaliteiten van de Proof of Concept.

\subsection{Positionering van het Onderzoek}

Ondanks de sterke bevestiging van de didactische effectiviteit van XR \autocite{Lampropoulos2025}, is er een tekortkoming in de literatuur met betrekking tot de gecombineerde en praktische implementatie van een kostenefficiënt, stand-alone XR-platform (Meta Quest 3), in combinatie met een schaalbare technische architectuur voor een brede curriculaire toepassing, en de specifieke vereisten voor CME-erkenning. Dit onderzoek pakt deze kloof aan door een toegepaste Proof of Concept te ontwikkelen die dient als gevalideerde blauwdruk voor toekomstige institutionele implementatie binnen het zorgonderwijs van HOGENT.
%---------- Methodologie ------------------------------------------------------
\section{Methodologie}%
\label{sec:methodologie}

Dit toegepaste onderzoek volgt in grote lijnen de stappen van de Design Science Research-methode. De kern ligt in het ontwikkelen en testen van een bruikbaar product, de Proof of Concept, als oplossing voor de opleidingsproblemen in de zorg. Het onderzoek is verdeeld in vier opeenvolgende fasen, die elk concrete resultaten opleveren en een deel van de onderzoeksvragen beantwoorden.

In de eerste fase worden de didactische, technische en inhoudelijke vereisten in kaart gebracht. Hierbij wordt gefocust op richtlijnen rond basisreanimatie, input van docenten verpleegkunde en inzichten uit de literatuur rond XR in het zorgonderwijs. Daarnaast worden bestaande CME-evaluatiekaders geanalyseerd met het oog op meetbaarheid en objectieve prestatie-evaluatie.

In de tweede fase wordt de Proof of Concept ontwikkeld. De applicatie simuleert één reanimatiescenario waarin de gebruiker geconfronteerd wordt met een noodsituatie. De XR-omgeving ondersteunt onder meer het herkennen van een onveilige situatie, bewustzijns- en ademhalingscontrole, het inroepen van hulp en het starten van de reanimatie. Tijdens het scenario worden objectieve prestatiegegevens gelogd, zoals volgorde van handelingen, timing en gemaakte fouten. Tegelijkertijd wordt er een Technische Architectuurstudie opgesteld om de gekozen opbouw te analyseren en te ontwerpen. Hierbij worden keuzes over de backend, de database voor prestatiedata en de mogelijkheden om het systeem in de toekomst uit te breiden met meer scenario's en meerdere gebruikers gemaakt. Dit beantwoordt de vraag naar de minimale technische voorwaarden voor schaalbaarheid. Het concrete eindproduct is een Functionele XR Proof of Concept op de Meta Quest 3, inclusief de bijbehorende backend. De XR-applicatie wordt ontwikkeld met de Unity 3D game engine, in combinatie met de C# programmeertaal.

Vervolgens richt de derde fase, Validatie, zich op het testen van de Proof of Concept in de praktijk. De Proof of Concept wordt gevalideerd via een kleinschalige pilottest of focusgroepen met de doelgroep van studenten en docenten Verpleegkunde van HOGENT. De validatie omvat het meten van hoe realistisch de ervaring is, de mate van immersie en de beoordeling van de onderwijskundige meerwaarde door docenten. Via dit gebruikersonderzoek wordt vastgesteld in welke mate de gebruikers de XR-ervaring als nuttig en gebruiksvriendelijk ervaren.
De gebruiksvriendelijkheid en immersie worden gekwantificeerd aan de hand van gestandaardiseerde instrumenten zoals de System Usability Scale (SUS) en een Likert-schaal voor de ervaren realiteitsgraad.

De afsluitende vierde fase, Strategische Roadmap, vertaalt de resultaten naar een toekomstplan. Op basis van de architectuuranalyse en de resultaten van de validatietest wordt een Roadmap opgesteld. Dit document beschrijft de technische en organisatorische stappen voor de gefaseerde uitrol en duurzame implementatie van een 'HOGENT XR Academy'. De roadmap zal een planning, budgettaire inschattingen en organisatorische aanbevelingen bevatten voor het verkrijgen van CME-accreditatie. De eindresultaten van deze fase zijn het Validatierapport en de Strategische Roadmap.

De totale uitvoering van dit onderzoek wordt ingeschat op een periode van 12 weken. De eerste fase (Vereisten) duurt de eerste drie weken. De tweede fase (Ontwikkeling van de POC) is gepland van week 4 tot en met week 8. De derde fase (Validatie) wordt uitgevoerd in de weken 9 en 10. Tot slot wordt de vierde en laatste fase (Strategische Roadmap en Conclusies) afgerond in de weken 11 en 12.
%---------- Verwachte resultaten ----------------------------------------------
\section{Verwachte Resultaten en Conclusies}%
\label{sec:verwachte_resultaten_en_conclusies}

Het belangrijkste resultaat is een functionele XR Proof of Concept die aantoont dat basisreanimatievaardigheden, inclusief contextuele en niet-technische aspecten, effectief geoefend kunnen worden binnen een XR-omgeving. Daarnaast levert het onderzoek inzicht op in de aard en bruikbaarheid van prestatiedata die door XR gegenereerd wordt, en in welke mate deze data aansluit bij bestaande CME-evaluatiecriteria.

De conclusies van deze bachelorproef zullen aantonen dat de XR SkillSim een haalbare en waardevolle aanvulling vormt op bestaande onderwijs- en trainingsmethoden, en dat het platform potentieel heeft om in de toekomst verder uitgebreid te worden met bijkomende verpleegkundige handelingen en scenario’s.

